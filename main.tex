\documentclass{article}

\usepackage[fancysections,tcbtheorems,enumerate,bibtex]{preamble}

\title{EECS16A}
\author{Jeremiah Vuong}
\date{2025-08-27}

\begin{document}

% \pagestyle{fancy}
% \fancyhead[L]{Game Theory}
% \fancyhead[R]{\thepage}

\maketitle

\tableofcontents
\newpage

\lecture[]{Introduction}
Quick intro via a quasi-lecture on image processing (?).

We can represent images via a linear combination of basis images.
$$
\vec{x} = \sigma_1 \vec{u}_1 + \ldots + \sigma_r \vec{u}_r
$$
Where $r$ is an incredibly large number.

NOTE: that if we were to order the basis images by their importance, we can represent the image with a much smaller number of basis images,
since the singular values $\sigma_i$ fall off logarithmically. (See: Singular value decomposition)

For optimization, we can approxomate (get a smaller combination):
$$ \vec{x} = \sigma_1 \vec{u}_1 + \ldots + \sigma_R \vec{u}_R $$
where $r >> R$.

Such that in fourier analysis, low frequency components (low detail) dominate.
In facial recognition, we need to go further down the combination since the beginning combinations
are features that most (all?) facies satisfy.


\end{document}

